
% source p2.1
\section{Algebraic semantics} \label{sec:source-2-1}
In this chapter we show that there is a generalised algebraic theory (the theory
of contextual categories) whose models (the category of contextual categories)
is equivalent to the category \underline{GAT} of generalised algebraic theories
and equivalences classos of interpretations.
%
How do we interpret this result?
%
Well, there are many other examples of this very strong kind of relationship
holding between an algebraic notion of structure and a syntactic notion of
theory.
%
The following list is by no means exhaustive:
\begin{table}
  \centering
 \begin{tabular}{lll}
  \underline{Syntactic Notion} & \underline{Algebraic Notion} & \underline{Reference}\\
   & &\\
  Proposition Theory\\
   of classical logic. & Boolean Algebra & \\
   & & \\
  Propositional Theory\\
   of Intuitionistic Logic. & Heyting Algebra. & \\
   & & \\
  Single Sorted Algebraic\\
   or Equational Theory. & Lawvere's Notion of an Algebraic Theory. & \cite[II]{Lawvere} \\
   & & \\
  Equational Theory in the language\\
   of the typed $\lambda$-calculus. & Cartesian Closed Category. & \cite[26]{Myeres}\\
   & & \\
  Theory of higher order\\
   Intuitionistic Logic. & Topos. &\cite{Fourman} \\
   & & \\
  Coherent Theory. & Grothendieck Site. & \cite[28]{Reyes}
\end{tabular}
\end{table}
\comment{citations needed in table - Can't read Fourman citation.}

%
% source p2.2
%

In all these cases there is definable the notion of a model of a given theory in
a given structure.
%
In each case the category of syntactic theories and equivalence classes of
interpretations is equivalent to the category of algebraic structures.
%
This last property is the important characterising property.
%
It can lead to the view that the theories in syntactic form should be dispensed
with entirely and the structures be given the title of theories.
%
This seems wasteful.
%
It is to be preferred that we think of the structures as providing a semantics
for the theories, in fact, the most general possible semantics.
%
We shall call it the algebraic semantics.
%
Thus contextual categories are to provide us with the algebraic semantics of
generalised algebraic theories.

In case it should be argued that what we have called the algebraic semantics is
really none other than the interpretations of one theory if considered as a
notion of semantics; well we more or less agree, though perhaps it is only when
such are considered as interpretations into algebraic structures that they can
be properly said to constitute a notion of semantics.
%
The important point here, though, is that structures do frequently appear quite
independently of theories; thus the notion of model is certainly enriched by the
isomorphism between theories and structures because theories which arise first as
structures (being defined by something like ``the theory that corresponds to
this here structure'') are usually theories which would not otherwise have
occurred.
\comment{ending of second line of second paragraph - ``i''?}
%
% source p2.3
%

\section{Definition and examples} \label{sec:source-2-2}





\lipsum[15]

% source p2.12
\section{Notation and basic lemmas} \label{sec:source-2-3}

\lipsum[16]

% source p2.24
\section{Contextual categories = generalised algebraic theories} \label{sec:source-2-4}

\lipsum[17]

% source p2.77
\section{Functorial semantics, universal algebra} \label{sec:source-2-5}
\comment{This section title is more capitalised in source; why?}

\lipsum[18]

%%% Local Variables:
%%% mode: latex
%%% TeX-master: "cartmell-thesis"
%%% End: 