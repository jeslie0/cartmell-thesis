\newcommand{\id}{\operatorname{id}}
% source p2.1
\section{Algebraic semantics} \label{sec:source-2-1}
In this chapter we show that there is a generalised algebraic theory (the theory of contextual categories) whose models (the category of contextual categories) is equivalent to the category \underline{GAT} of generalised algebraic theories and equivalences classos of interpretations.
%
How do we interpret this result?
%
Well, there are many other examples of this very strong kind of relationship holding between an algebraic notion of structure and a syntactic notion of theory.
%
The following list is by no means exhaustive:
\begin{table}
  \centering
 \begin{tabular}{lll}
  \underline{Syntactic Notion} & \underline{Algebraic Notion} & \underline{Reference}\\
   & &\\
  Proposition Theory & Boolean Algebra\\
   of classical logic. \\
   & & \\
  Propositional Theory & Heyting Algebra\\
   of Intuitionistic Logic. \\
   & & \\
  Single Sorted Algebraic & Lawvere's Notion of\\
   or Equational Theory. & an Algebraic Theory. & \cite[II]{Lawvere} \\
   & & \\
  Equational Theory in the language & Cartesian Closed\\
   of the typed \(\lambda\)-calculus. & Category. & \cite[26]{Myeres}\\
   & & \\
  Theory of higher order & Topos.\\
   Intuitionistic Logic. &  &\cite{Fourman} \\
   & & \\
  Coherent Theory. & Grothendieck Site. & \cite[28]{Reyes}
\end{tabular}
\end{table}
\comment{citations needed in table - Can't read Fourman citation.}

%
% source p2.2
%

In all these cases there is definable the notion of a model of a given theory in a given structure.
%
In each case the category of syntactic theories and equivalence classes of interpretations is equivalent to the category of algebraic structures.
%
This last property is the important characterising property.
%
It can lead to the view that the theories in syntactic form should be dispensed with entirely and the structures be given the title of theories.
%
This seems wasteful.
%
It is to be preferred that we think of the structures as providing a semantics for the theories, in fact, the most general possible semantics.
%
We shall call it the algebraic semantics.
%
Thus contextual categories are to provide us with the algebraic semantics of generalised algebraic theories.

In case it should be argued that what we have called the algebraic semantics is really none other than the interpretations of one theory if considered as a notion of semantics; well we more or less agree, though perhaps it is only when such are considered as interpretations into algebraic structures that they can be properly said to constitute a notion of semantics.
%
The important point here, though, is that structures do frequently appear quite independently of theories; thus the notion of model is certainly enriched by the isomorphism between theories and structures because theories which arise first as structures (being defined by something like ``the theory that corresponds to this here structure'') are usually theories which would not otherwise have occurred.
\comment{ending of second line of second paragraph - ``i''?}
%
% source p2.3
%

\section{Definition and examples} \label{sec:source-2-2}

\begin{definition}
  A \defemph{contextual category} consists of
  \begin{enumerate}
  \item A category \(\underline{C}\) with a terminal object 1.
  \item A tree structure on the objects of \(\underline{C}\) such that the terminal object 1 is the unique least element of the tree. 
  \item For all \(A, B \in | \underline{C} |\) such that \(A \triangleleft B\) a morphism \(p(B) : B \rightarrow A\) in \(\underline{C}\). This morphism will also be written just as \(B \rightarrowtriangle A\).
  \item For all \(A, A' \in |\underline C |\), for all \(f:A \rightarrow A'\) in \(\underline{C}\), for all \(B \in |\underline C |\) such that \(A' \triangleleft B\), an object \(f^*B\) of \(\underline C\) and morphism \(q(f,B) : f^*B \rightarrow B\) such that \(A \triangleleft f^*B\) and such that the diagram
    \begin{center}
      \begin{tikzcd}
        f^*B \arrow[r, "{q(f,B)}"] \arrow[d] & B \arrow[d,] \\
        A \arrow[r, "f"']                    & A'         
      \end{tikzcd}
    \end{center}

    is a pullback diagram in \(\underline C\).
  \end{enumerate}

  Such that
  \begin{enumerate}[(i)]
    \item For all \(A, B \in | \underline C|\) such that \(A \triangleleft B\), \(\id_A^*B = B\) and \(q(\id_A,B) = \id_B\).
    \item Whenever
      \begin{center}
        \begin{tikzcd}
          &                     & B \arrow[d] \\
          A \arrow[r, "f"'] & A' \arrow[r, "f'"'] & A''        
        \end{tikzcd}
      \end{center}
      in \(\underline C\) then \((ff')^*B = f^*(f'^*B)\) and \(q(ff',B) \circ q(f',B)\).
  \end{enumerate}
\end{definition}
\comment{Need to add triangle heads to arrows in diagrams.}
%
% source p2.4
We shall see that the objects of a contextual category should be though of as contexts.
%
Recall that a context is a sequence \( \langle x_1 \in \Delta_1, \ldots, x_n \in \Delta_n \rangle\) such that the rule \(x_1 \in \Delta_1, \ldots, x_{n-1} \in \Delta_{n-1} : \Delta_n\) is a type    is a derived rule and such that \(x_n\) is a variable distinct from each \(x_1, \ldots, x_{n-1}\).
% added a comma after the \ldots. Unsure how to go about making spacing work for'' `is a type'.
The tree structure on the set of contexts of a theory is easily seen.
%
For \(n \geq 1\), the predecessor of a context \(\langle x_1 \in \Delta_1, \ldots, x_n \in \Delta_n \rangle\) is the context \(\langle x_1 \in \Delta_1, \ldots, x_{n-1} \in \Delta_{n-1} \rangle\).
%
The empty context \(\langle  \rangle\) is the unique least element of the tree.

The morphisms of a contextual category should be thought of as realisations. Recall that a realisation of a context \(\langle y_1 \in \Omega_1, \ldots, y_m \in \Omega_m \rangle\) with respect to the context $\langle  x_1 \in \Delta_1, \ldots, x_n \in \Delta_n \rangle$ is just an \(m\) tuple \(t_1, \ldots, t_m\) such that for each \(j\), \(1 \leq j \leq m\), the rule \(x_1 \in \Delta_1, \ldots, x_n \in \Delta_n : t_j \in \Omega_j [t_1|y_1, \ldots, t_{j-1}|y_{j-1}]\) is a derived rule.
%
Think of the morphisms \(f:A \rightarrow A'\) in a contextual category as being a realisation of the context \(A'\) wtr the context A.

In \ref{Section:1.12} we defined the category \(\underline{R(U)}\) of contexts and realisations of a theory \(U\), \comment{Capital ``W'' in ``we''?} we could go on and show that for any theory \(U\), the category \(\underline{R(U)}\) with pullback structure defined in \ref{Section:1.12} (actually we defined more structure than was necessary) is a contextual category.\comment{no full stop here, but seems like end of sentence}
%
But we do not need this construction.
%
Rather we need the construction of a contextual category \(\mathbb C (U)\) associated with a theory \(U\) as part of the equivalence between contextual categories and generalised algebraic theories.
%
This category \(\mathbb C (U)\) is a category of equivalence classes of contexts and equivalence classes of realisations of \(U\).
%
We shall describe in some detail.

Recall that in \ref{Section:1.13} we defined an equivalence relation \(\equiv\) on derived \(T\) and \(\in\)-rules of a theory \(U\).
%
% source p2.5
%
This equivalence relation we call the equivalence relation of intended identity of denotation.
%
We used this equivalence relation in definin an equivalence relation \(\equiv\) on contexts and realisations: \(\langle x_1 \in \Delta)1, \ldots, x_n \in \Delta_n\rangle \cong \langle y_1 \in \Omega_1, \ldots, y_m \in \Omega_m \rangle\) iff \(\inferrule{x_1 \in \Delta_a, \ldots, x_{n-1} \in \Delta_{n-1}} {\Delta \text{ is a type}} \equiv \inferrule{y_1 \in \Omega_1, \ldots, y_{m-1} \in \Omega_{m-1}}{\Omega_m \text{ is a type}}\).\comment{maybe put on one or two lines?}
%
And if \(\langle t_1, \ldots, t_m \rangle\) is a realisation of \(\langle y_1 \in \Omega_1, \ldots, y_m \in \Omega_m \rangle\) wrt \(\langle x_1 \in \Delta_1, \ldots, x_n \in \Delta_n \rangle\) and if \(\langle t'_1, \ldots, t'_m \rangle\) is a realisation of \(\langle y'_1 \in \Omega'_1, \ldots, y'_m \in \Omega'_m \rangle\) wrt \(\langle x'_1 \in \Delta'_1, \ldots, x'_n \in \Delta'_n \rangle\) then \(\langle t_1, \ldots, t_m \rangle \equiv \langle t'_1, \ldots, t'_m \rangle\) iff for each \(j\), \(1 \leq j \leq m\),
%
\[\inferrule{x_1 \in \Delta_1, \ldots, x_n \in \Delta_n}{t_j \in \Omega_j[t_1 | y_1, \ldots t_{j-1}|y_{j-1}]} \equiv \inferrule{x'_1 \in \Delta'_1, \ldots, x'_n \in \Delta'_n}{t'_j \in \Omega'_j[t'_1 | y'_1, \ldots t'_{j-1}|y'_{j-1}]}\]
%
The category \(\mathbb C (U)\) is defined to have as objects the equivalence classes of contexts of \(U\) and to have as morphisms the equivalence classes of realisations of \(U\).
%
More precisely if \(\langle x_1 \in \Delta_1, \ldots, x_n \in \Delta_n \rangle\) and \(\langle y_1 \in \Omega_1, \ldots, y_m \in \Omega_m \rangle\) are contexts of \(U\) then define \(\text{Hom}\mathbb C  (U)(\langle x_1 \in \Delta_1, \ldots, x_n \in \Delta_n \rangle, \langle y_1 \in \Omega_1, \ldots, y_m \in \Omega_m \rangle) = \{[\langle  t_a, \ldots, t_m \rangle \mid \langle t_1, \ldots, t_m  \rangle] \text{ is a relisation of } \langle y_a \in \Omega_a, \ldots, y_m \in \Omega_m \rangle \text{ wtr } \langle x_1 \in \Delta_1, \ldots, x_n \in \Delta_n \rangle\}\). Hom is well defined just by Lemma \ref{4(i)} of \ref{Section:1.13}.
%
Whenever \(\langle t_1, \ldots, t_m \rangle\) is a realisation of \(\langle y_a \in \Omega_1, \ldots, y_m \in \Omega_m \rangle\) and whenever \(\langle s_1, \ldots, s_l \rangle\) \comment{Unsure on final s subscript for this part.} is a realisation wrt \(\langle y_1 \in \Omega_1, \ldots, y_m \in \Omega_m \rangle\) then the composition in \(\mathbb{C (U)}\) of \([\langle t_1, \ldots, t_m \rangle]\) with \([\langle s_1, \ldots, s_l \rangle]\) is defined by \([\langle t_1, \ldots, t_m \rangle] \circ [\langle s_1, \ldots, s_l \rangle] = [\langle s_1[t_1|y_1, \ldots, t_m|y_m], \ldots, s_l[t_1|y_1, \ldots t_m|y_m] \rangle]\).
%
Composition is well defined, this follows from Lemma \ref{lem:3(ii)} of \ref{section:1.13}. The identity morphisms in \(\mathbb C (U)\) are given by $\id_{[\langle x_1 \in \Delta_1, \ldots, x_n \in \Delta_n \rangle]} = [\langle x_1, \ldots, x_n \rangle]$.
%
Well definedness is trivial.
%
% source p2.6
% 
The objects of \(\mathbb C (U)\) are structured as a tree by taking the predecessor of \([\langle x_1 \in \Delta_1, \ldots x_n \in \Delta_n \rangle]\) to be \([\langle x_1 \in \Delta_1, \ldots x_{n-1} \in \Delta_{n-1} \rangle]\).
%
The tree structure is well defined because by definition if \(\langle x_1 \in \Delta_1, \ldots x_n \in \Delta_n \rangle \equiv \langle x_1' \in \Delta_1', \ldots x_n' \in \Delta_n' \rangle\) then \(\langle x_1 \in \Delta_1, \ldots x_{n-1} \in \Delta_{n-1} \rangle \equiv \langle x_1' \in \Delta_1', \ldots x_{n-1}' \in \Delta_{n-1}' \rangle\).
%
\([\langle  \rangle]\) is the least element of the tree.
%
\([\langle  \rangle]\) is a terminal object of \(\mathbb C(U)\) because by definition of \(\text{Hom}\mathbb C (U)\), \(\text{Hom}\mathbb C (U) ([\langle x_1 \in \Delta_1, \ldots, x_n \in \Delta_n \rangle], [\langle  \rangle]) = \{[\langle  \rangle] \mid \langle  \rangle \text{ is a relisation of } \langle  \rangle \text{ wrt } \langle x_1 \in \Delta_1, \ldots, x_n \in \Delta_n \rangle\}\) and because by definition of realisation, \(\langle \rangle\) is the realisation of the context \(\langle \rangle\) wrt the context \( \langle x_1 \in \Delta_1, \ldots, x_n \in \Delta_n \langle \).

If \(A \triangleleft B\) in \(\mathbb C (U)\), say \(A = [\langle x_1 \in \Delta_1, \ldots x_n \in \Delta_n \rangle]\) and \(B = [\langle x_1 \in \Delta_1, \ldots x_n \in \Delta_n, x \in \Delta \rangle]\), then define \(p(B):B \rightarrow A\) by \(p(B) = [\langle x_1, \ldots, x_n \rangle]\).

If
\begin{center}
  \begin{tikzcd}
    & B \arrow[d]\\
    A \arrow[r, "f"] & A'
  \end{tikzcd}\comment{Fix vertical arrow head}
\end{center}

in \(\mathbb C (U)\), say \(A = [\langle x_1 \in \Delta_1, \ldots, x_n \in \Delta_n \rangle]\), \(A' = [\langle y_1 \in \Omega_1, \ldots, y_m \in \Omega_m \rangle]\), \(B = [\langle y_1 \in \Omega_1, \ldots, y_m \in \Omega_m, y \in \Omega \rangle]\) and \(f = [\langle t_1, \ldots, t_m \rangle]\), then define \(f^*B=[\langle x_1 \in \Delta)1, \ldots, x_n \in \Delta_n, y \in \Omega[t_1|y_1, \ldots t_m|y_m] \rangle]\) and \(q(f,B) = [\langle t_1, \ldots, t_m, y \rangle]\).
%
\(f^*B\) is well defined, by Lemma \ref{3i, section 1.13}. 

\begin{lemma}
  \(\mathbb C (U)\) is a contextual category.
\end{lemma}
%
% source p2.7
%

\begin{proof}
  Firstly we must show that whenever
    \begin{center}
      \begin{tikzcd}
        & B \arrow[d]\\
        A \arrow[r] & A'
      \end{tikzcd}\comment{Fix vertical arrow head}
    \end{center}

    in \(\mathbb C(U)\) then the diagram
    \begin{center}
      \begin{tikzcd}
        f^*B \arrow[d] \arrow[r, "{q(f,B)}"] & B \ar[d] \\
        A \ar[r, "f"'] & A'
      \end{tikzcd}
    \end{center}\comment{Fix vertical arrow head}

    is a pullback diagram in \(\mathbb C(U)\).

    So suppose that \(A = [\langle x_1 \in \Delta_1, \ldots, x_n \in \Delta_n \rangle]\), \(B = [\langle y_1 \in \Omega_1, \ldots, y_m \in \Omega_m, y \in \Omega \rangle]\) and \(f = [\langle t_1, \ldots t_m \rangle]\). Suppose also that \(C\) is an object of \(\mathbb C(U)\) and that \(g:C \rightarrow A\) and \(g':C \rightarrow B\) in \(\mathcal C(U)\) such that the diagram

    \begin{center}
      \begin{tikzcd}
        C \arrow[d, "g"'] \arrow[r, "g"'] & B \ar[d] \\
        A \ar[r, "f"] & A'
      \end{tikzcd}
    \end{center}
    commutes. Call this diagram (I). We can suppose that \(C = [\langle z_1 \in \Lambda_1, \ldots, z_p \in \Lambda_p \rangle]\), \(g = [\langle r_1, \ldots, r_n \rangle]\), \(g' = [\langle s_1, \ldots, s_m, s \rangle]\), where \(\langle r_1, \ldots, r_n \rangle\) is some realisation of \(\langle x_1 \in \Delta_1, \ldots, x_n \in \Delta_n \rangle\) wrt \(\langle z_1 \in \Lambda_1, \ldots, z_p \in Lambda_p \rangle\) and \(s_1, \ldots, s_m, s\) is a realisation of \(\langle y_1 \in \Omega_1, \ldots, y_m \in \Omega, y \in \Omega \rangle\) wrt \(\langle z_1 \in \Lambda_!, \ldots, z_p \in \Lambda_p \rangle\).
    We must show that there exists a unique \(h:C \rightarrow f^*B\) in \(\mathbb C(U)\) such that diagrams \ref{diag:2.7-II} and \ref{diag:2.7-III} both commute.

    \begin{figure}
    \centering
    \begin{subfigure}{.5\textwidth}
      \centering
      \begin{tikzcd}
          C \arrow[rd, "h"] \arrow[rdd, "g"'] &                \\
          & f^*B \arrow[d] \\
          & A             
        \end{tikzcd}
      \caption{II}
      \label{diag:2.7-II}
    \end{subfigure}%
    \begin{subfigure}{.5\textwidth}
      \centering
      \begin{tikzcd}
        C \arrow[rd, "h"'] \arrow[rrd, "g'"] &                             &   \\
        & f^*B \arrow[r, "{q(f,b)}"'] & B
      \end{tikzcd}
      \caption{III}
      \label{diag:2.7-II}
    \end{subfigure}
    \caption{A figure with two subfigures}
    \label{fig:test}
  \end{figure}\comment{Float position needs to be fixed}

  I claim that \([\langle r_1, \ldots, r_m s \rangle]\) is such an \(h\).
  %
  Since Diagram (I)\comment{What diagram is this?} commutes, \([\langle t_![r_1|x_1, \ldots, r_n|x_n], \ldots, t_m[r_1|x_1, \ldots, r_n|x_n] \rangle] = [\langle s_1,\ldots, s_m \rangle]\).
  %
  Hence for all \(j\), \(1 \leq j \leq m\),

  \[
    \inferrule{z_1 \in \Lambda_1, \ldots, z_p \in \Lambda_p}{t_1[r_1|x_1, \ldots, r_n|x_n], \ldots, t_m[r_1|x_1, \ldots, r_n|x_n]}
  \]\comment{conclusion is missing from document - I assume this is what it says.}

  is a derived rule of \(U\).
  %
  % Source p2.8
  %
  Thus as \(\langle  s_!, \ldots, s_m, s \rangle\) is a realisation of \(\langle y_1 \in \Omega_1, \ldots, y_m \in \Omega_m, y \in \Omega \rangle\), the rule 
  \[
    \inferrule{z_1 \in \Lambda_1, \ldots, z_p \in \Lambda_p}{s \in \Omega [t_1[r_1|x_1, \ldots, r_n|x_n]|y_1, \ldots, t_m[r_1|x_1, \ldots, r_n|x_n]|y_m}
  \]
  is a derived rule of \(U)\).
  %
  Hence \(\langle r_1,\ldots,r_n, s \rangle\) is a realisation of \(\langle x_1 \in \Delta_1, \ldots, x_n \in \Delta_n, y \in \Omega[t_1|y_1, \ldots, t_m|y_m] \rangle\) wrt \(\langle z_1 \in \Lambda_1, \ldots, z_p \in \Lambda_p \rangle\) and thus \([\langle r_1, \ldots, r_n,s \rangle] : C \rightarrow f^*B\) in \(\mathbb C (U)\).
  %
  Setting \(h = [\langle  r_1, \ldots, r_n, s \rangle]\) then \ref{diag:2.7-II} commutes because \([\langle r_1, \ldots, r_m, s \rangle] \circ [\langle t_1, \ldots, t_m,y \rangle] = [\langle t_1[r_1|x_1, \ldots, r_n|x_n], \ldots, t_m[r_1|x_1, \ldots, r_n|x_n],s \rangle] = [\langle s_1, \ldots, s_m,s \rangle]\).
  %
  So \([\langle r_1, \ldots, r_m, s \rangle]\) is certainly such an \(h\).
  %
  To show that it is the unique such \(h\) suppose now that \(h\) is an arbitrary morphism \(h:C \rightarrow f^*B\) in \(\mathbb C(U)\) such that the diagrams \ref{diag:2.7-II} and \ref{diag:2.7-II} commute say \(h = [\langle r_1', \ldots, r_n', s' \rangle]\).
  %
  Since \ref{diag:2.7-II} commutes, for each \(i\), \(1 \leq i \leq n\), the rule \(\inferrule{z_1 \in \Lambda_1, \ldots, z_p \in \Lambda_p}{r_i = r_i' \in \Delta_i[r_1|x_1, \ldots, r_{i-1}|x_{i-1}]}\) is a derived rule of \(U\).
  %
  Since \ref{diag:2.7-III} commutes, the rule \(\inferrule{z_1 \in \Lambda_1, \ldots, z_p \in \Lambda_p}{s=s' \in \Omega[s_1|y_1, \ldots,s_m|y_m]}\) is a derived rule of \(U\).
  %
  Hence \(h = [\langle r_1', \ldots, r_n', s' \rangle] = [\langle r_1, \ldots, r_n, s \rangle]\).
  %
  Which completes the proof that
  \begin{center}
    \begin{tikzcd}
      f^*B \arrow[r, "{q(f,B)}"] \arrow[d] & B \arrow[d] \\
      A \arrow[r, "f"] & A'
    \end{tikzcd}\comment{fix vertical arrow heads}
  \end{center}
  is a pullback diagram in \(\mathbb C(U)\).

  It remains to show that the axioms (I) and (II)\comment{fix reference} of the definition of a contextual category hold of the structure \(\mathbb C(U)\).
  %
  % Source p2.9
  %
  Well, it is easy to show that (I)\comment{fix reference} must hold because if \(B=[\langle  x_1 \in \Delta_1, \ldots, x_n \in \Delta_n, x \in \Delta \rangle]\) then \(\Delta[x_1|x_1, \ldots, x_n|x_n] = \Delta\).
  %
  Similarly (III) \comment{fix reference} holds because \(\Lambda[s_1 | z_1, \ldots, s_p|z_P] [ t_1|y_1, \ldots, t_m|y_m] = \Lambda[s_1[t_1|y_1, \ldots, t_m|y_m]    s_p[t_1|y_1, \ldots, t_m|y_m][z_p]]\)\comment{part of equation off page - unsure on brackets}, whenever \(\langle z_1 \in \Lambda_1, \ldots, z_p \in \Lambda_p, z \in \Lambda \rangle\) is a context and \(\langle  s_1, \ldots, s_p \rangle\) is a realisation of \(\langle z_1 \in \Lambda_1, \ldots, z_p \in \Lambda_P \rangle\) wrt \(\langle y_1 \in \Omega_1, \ldots, y_m \in \Omega_m \rangle\) and \(\langle t_1, \ldots, t_m \rangle\) is a realisation of \(\langle y_1 \in \Omega_1, \ldots, y_m \in \Omega_m \rangle\).
\end{proof}

We now turn to the definition of a (large) contextual category, \(\catFam\), which plays the same role among contextual categories as does the category \(\catSet\) among categories.
%
Whereas \(\catSet\) is the structured collection of functions so it is that \(\catFam\) is the structured collection of operators. .
%
We must refer back to \ref{section:1.9}\comment{fix ref} to the discussion of operators and sets, families of sets, families of families of sets and so on.

The tree object of \(\catFam\) is the tree of families introduced in \ref{section:1.9}.
%
Thus it is the tree of sets, families of sets, families of families of sets and so on with a formally adjoined least element 1.

For \(n, m \geq 0\), if \(1 \triangleleft A_1 \triangleleft \ldots \triangleleft A_n\) and \(1 \triangleleft B_1 \ldots \triangleleft B_m\) in \(\catFam\) then \(\Hom_\catFam(A_n, B_M) \underset{def}{=} \{\langle f_1, \ldots, f_m \rangle \mid f_1, \ldots, f_m \text{ are operators such that the status of the operator } f_j \text{ is given by for } a_1 \in A_1, \text{ for } a_2 \in A_2(a_1), \ldots, \text{ for } a_n \in A_n(a_1, \ldots, a_{n-1}):f_j(a_1, \ldots, a_n) \in B_j(f_1(a_1, \ldots, a_n), \ldots, f_{j-1}(a_1, \ldots, a_n))\}\).

In particular if \(n=0\) then we get \(\Hom(1, B_m) = \{\langle b_1, \ldots, b_m \rangle \mid b_1 \in B_1, b_2 \in B_2(b_1), \ldots, \text{ and } b_m \in B_M(b_1, \ldots, b_{m-1})\}\).
%
On the other hand \(1\) is the terminal object of \(\catFam\) because \(\Hom(A_n, 1) = \{\langle  \rangle\}\).
%
% Source p2.10
%










\lipsum[15]

% source p2.12
\section{Notation and basic lemmas} \label{sec:source-2-3}

\lipsum[16]

% source p2.24
\section{Contextual categories = generalised algebraic theories} \label{sec:source-2-4}

\lipsum[17]

% source p2.77
\section{Functorial semantics, universal algebra} \label{sec:source-2-5}
\comment{This section title is more capitalised in source; why?}

\lipsum[18]

%%% Local Variables:
%%% mode: latex
%%% TeX-master: "cartmell-thesis"
%%% End: 