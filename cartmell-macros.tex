% Contents:
%
% - Binary relations
% - Category names
% - Single letters

%%%%
% Semantic markers for text
%%%%

\newcommand{\defemph}{\emph} % (possibly) emphasise a definiendum

%%%%
% Binary relations, operators
%%%%

%%%% 
% Single styled characters (or almost single) and character-like symbols
%%%%

\newcommand{\catTwo}{\mathbf{2}} % used for the arrow category

\newcommand{\catC}{\catletter{C}}
\newcommand{\catD}{\catletter{D}}
\newcommand{\modM}{\mathcal{M}}
\newcommand{\varsV}{\mathcal{V}}
\newcommand{\thU}{\mathcal{U}}

%%%%
% Styled words: general
%%%%

% Categories
\newcommand{\catletter}[1]{\mathbf{#1}} % single-letter categories
\newcommand{\catword}[1]{\mathbf{#1}} % long-named categories

\newcommand{\catCat}{\catword{Cat}}
\newcommand{\catFam}{\catword{Fam}}
\newcommand{\catGAT}{\catword{GAT}}
\newcommand{\catSet}{\catword{Set}}
\newcommand{\catLEX}{\catword{LEX}}
\newcommand{\catCon}{\catword{Con}}

\DeclareMathOperator{\proj}{proj}
%%%%
% Styled words: type theory syntax
%%%%

\newcommand{\syn}[1]{\mathsf{#1}}

% Standard environment for specifying a theory
\newenvironment{theoryspec}
  {\begin{tabular}{ll}  Symbol. & Introductory Rule. \\}
  {\end{tabular}}
\newcommand{\axioms}{ \multicolumn{2}{l}{Axioms.} \\ }
\newcommand{\oneaxiom}{ \multicolumn{2}{l}{Axiom.} \\ }
\newcommand{\noaxioms}{ \multicolumn{2}{l}{Axioms---None.} \\ }
\newcommand{\axiom}[1]{\multicolumn{2}{l}{#1} \\}

% Parts of judgements
\newcommand{\isatype}{\text{ is a type}}

% Variables and metavariables: just use normal math font.

% Constants of specific theories

% Extensions by disjoint unions
\newcommand{\synSum}{\mathop{\Sigma}\displaylimits} % TODO: sans serif sigma?
\newcommand{\synPr}{\syn{Pr}}

% Extensions by equality predicates
\newcommand{\synEq}{\syn{Eq}}
\newcommand{\synr}{\syn{r}}

% Categories and functors
\newcommand{\synOb}{\syn{Ob}}
\newcommand{\synHom}{\syn{Hom}}
\newcommand{\synid}{\syn{id}}
\newcommand{\syno}{\syn{o}}
\newcommand{\synF}{\syn{F}}
\newcommand{\synMorph}{\syn{Morph}}
\newcommand{\syncod}{\syn{cod}}
\newcommand{\syndom}{\syn{dom}}

%%%%
% Other operators
%%%%

\newcommand{\tuple}[1]{\langle #1 \rangle}
\newcommand{\disjointunion}{\mathop{\dot{\bigcup}}}
\newcommand{\Hom}{\operatorname{Hom}}

%%%%
% Other symbols
%%%%

\newcommand{\suchthat}{\ |\ } % for use in set comprehensions: \{ x \suchthat P(x) \} 

% Arrows
\newcommand{\imp}{\leftrightarrow} % logical implication

%%%%
% Editing notes
%%%%

\newcommand{\comment}[1]{\marginpar{\textcolor{red}{#1}}}
\newcommand{\placeholder}[1]{\textcolor{red}{#1}}

%%% Local Variables:
%%% mode: latex
%%% TeX-master: "cartmell-thesis"
%%% End: 